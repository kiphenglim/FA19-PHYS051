\documentclass[11pt,letterpaper,boxed, noheader]{pset}

\usepackage[margin=0.75in]{geometry}
\usepackage{ulem}

\begin{document}

    \problemlist{PHYS051 HW13}
    \begin{center}
        T1.4; T1.9; T1.13
    \end{center}
    
    \begin{problem} [T1.4]
        A radio station broadcasts at a frequency $v = 91.5$ MHz with a total radiated power of $P = 20$ kW. 
        
        \begin{enumerate}
            \item [a.] What is the wavelength $\lambda$ of this radiation?
            \item [b.] What is the energy of each photon in eV? How many photons are emitted each second? How many photons are emitted in each cycle?
            \item [c.] A particular radio receiver requires 2.0 microwatts of radiation to provide intelligible reception. How many 91.5 MHz photons does this require per second? per cycle?
            \item [d.] Do the answers to \textit{(b)} and \textit{(c)} indicate that the granularity of the electromagnetic radiation can be neglected in these circumstances?
        \end{enumerate}
    \end{problem}
    \newpage
    
    \begin{problem} [T1.9]
        A beam of UV light of wavelength $\lambda = 197.0$ nm falls onto a metal cathode. The stopping potential needed to keep any electrons from reaching the anode is 2.08 V. 
        
        \begin{enumerate}
            \item [a.] What is the work function $W$ of the cathode surface, in eV?
            \item [b.] What is the velocity $v$ of the fastest electrons emitted from the cathode? \textit{Note:} Since $K_{max}/mc^2 \ll 1$, the nonrelativistic expression for this kinetic energy can be utilized here.
            \item [c.] If Avogadro's number of photons strikes each square meter of the surface in one hour, what is the average intensity $I$ of the beam, in units of $W/m^2$?
        \end{enumerate}
    \end{problem}
    \newpage
    
    \begin{problem} [T1.13]
        The maximum kinetic energy of electrons ejected from sodium is 1.85 eV for radiation of 300 nm and 0.82 eV for radiation of 400 nm. Use this data to determine Planck's constant and the work function of sodium.
    \end{problem}
\end{document}