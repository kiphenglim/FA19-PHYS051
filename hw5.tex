\documentclass[11pt,letterpaper,boxed]{pset}

\usepackage[margin=0.75in]{geometry}
\usepackage{ulem}

\begin{document}

    \problemlist{PHYS051 HW05}
    \begin{center}
        P29.6, *E31.48
    \end{center}
    
    \begin{problem} [P29.6]
    \begin{itemize}
        \item [(a)] The current density across a cylindrical conductor of radius $R$ varies according to the equation 
        \[j = j_0 (1 - \frac{r}{R})\]
        where $r$ is the distance from the axis. Thus the current density is a maximum $j_0$ at the axis $r=0$ and decreases linearly to zero at the surface $r=R$. Calculate the current in terms of $j_0$ and the conductor's cross-sectional area $A=\pi R^2$.
        \item [(b)] Suppose that, instead, the current density is a maximum $j_0$ at the surface and decreases linearly to zero at the axis, so that
        \[j = j_0\frac{r}{R}\]
        Calculate the current. Why is the result different from \textit{(a)}?
    \end{itemize}
    \end{problem}
    \newpage
    
    \begin{problem} [*E31.48]
    A 1.0-$\mu$F capacitor with an initial stored energy of 0.50 J is discharged through a 1.0-M$\Omega$ resistor. 
    
    \begin{itemize}
        \item [(a)] What is the initial charge on the capacitor?
        \item [(b)] What is the current through the resistor when the discharge starts?
        \item [(c)] Determine $\Delta V_C$, the voltage across the capacitor, and $\Delta V_R$, the voltage across the resistor, as functions of time.
        \item [(d)] Express the rate of generation of internal energy in the resistor as a function of time.
    \end{itemize}
    \end{problem}
    \newpage
\end{document}