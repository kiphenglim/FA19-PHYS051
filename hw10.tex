\documentclass[11pt,letterpaper,boxed]{pset}

\usepackage[margin=0.75in]{geometry}
\usepackage{ulem}

\begin{document}

    \problemlist{PHYS051 HW10}
    \begin{center}
        P35.1, E35.12
    \end{center}
    
    \begin{problem} [P35.1]
        A thin, plastic disk of radius $R$ has a charge $q$ uniformly distributed over its surface. If the disk rotates at an angular frequency $\omega$ about its axis, show that magnetic dipole moment of the disk is 
        
        \[\mu = \frac{\omega qR^2}{4}.\]
        
        (Hint: The rotating disk is equivalent to an array of current loops.)
    \end{problem}
    \newpage
    
    \begin{problem} [E35.12]
        The dipole moment associated with an atom of iron in an iron bar is 2.22 $\mu_B$. Assume that all the atoms in the bar, which is 4.86 cm long and has a cross-sectional area of 1.31 cm$^2$, have their dipole moments aligned.
        
        \begin{enumerate}
            \item [a.] What is the dipole moment of the bar?
            \item [b.] What torque must be exerted to hold this magnet at right angles to an external field of 1.53 T?
        \end{enumerate}
    \end{problem}
\end{document}